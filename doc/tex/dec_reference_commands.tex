%
% Elecnor Deimos
%
% Data Exchange Component
%
% Borja Lopez Fernandez (BOLF)
%
% DEC Reference commands
%

\documentclass[dec_sum_main.tex]{subfiles}

\newcommand{\sectionbreak}{\clearpage}
 

 
\begin{document}

\label{decValidateConfig}
\section{decValidateConfig}

\begin{verbatim}

== Synopsis
	
This is a DEC command line tool that checks the validity of DEC configuration
files according to DEC's XSD schemas. This tool should be run everytime a 
configuration change is performed.

-e flag:

With this option the Interfaces (Entities) configuration file (dec_interfaces.xml)
is validated using the schema interfaces.xsd

-g flag:

With the main DEC configuration file (dec_config.xml)
is validated using the schema dec_config.xsd

-i flag:

With this option the Incoming file-types configuration file (dec_incoming_files.xml)
is validated using the schema dec_incoming_files.xsd

-o flag:

With this option the Outgoing file-types configuration file (dec_outgoing_files.xml)
is validated using the schema dec_outgoing_files.xsd

-m flag:

With this option the DEC Mail configuration file (ft_mail_config.xml) is
validated using the schema ft_mail_config.xsd

-l flag:

With this option the DEC Logs configuration file (dec_log_config.xml) is
validated using the schema dec_log_config.xsd

-a flag:

This is the all flag, which performs all the checks described before.


== Usage
-a          Check all DEC configuration files
-g          Check DEC's general configuration file dec_config.xml
-e          Check the Entities Configuration file dec_interfaces.xml
-m          Check the mail configuration file ft_mail_config.xml
-i          Check the incoming file-types configuration file dec_incoming_files.xml
-i          Check the outgoing file-types configuration file dec_outgoing_files.xml
-X <dir>    eXtract the configuration into the specified directory
-N <label>  label of the node to be appended into the configuration
-h          shows this help
-v          shows version number


== Author
DEIMOS-Space S.L.

== Copyright
Copyright (c) 2007 ESA - DEIMOS Space S.L
	
\end{verbatim}


\label{decCheckConfig}
\section{decCheckConfig}
 
\begin{verbatim}
== Synopsis

This is a command line tool that checks the coherency of the DEC configuration.
DEC configuration is distributed amongst different XML files. The information set up
must be coherent. This tool ensures that all configuration critical elements are correct.
(All DEC config files must be placed in the $DEC_CONFIG directory). 
So, run this tool everytime a configuration change is performed.

-e flag:

With this option the Interfaces (Entities) configuration placed in dec_interfaces.xml
is checked. As well it is checked the coherency between the dec_interfaces.xml
configuration file and the DEC Inventory (DEC Database).
(Note: if the network link to a given I/F is broken, the tool will not be able to connect and it
will report a configuration error of this I/F).


-i flag:

With this option the Incoming file-types registered in the dec_incoming_files.xml are checked.
Mainly what it is done is to check that the interfaces from a File is retrieved are configured in 
the dec_interfaces.xml file.


-m flag:

With this option the DEC Mail configuration placed in the ft_mail_config.xml is checked.


-s flag:

With this option the DCC Services configured in the dcc_services.xml file are checked.
The check performed with this flag is that the executable set in the command of the service
can be found in the $PATH environment variable.


-t flag:

With this option the In-Trays configured in the dec_incoming_files.xml file are checked.


-a flag:

This is the all flag, which performs all the checks described before.


== Usage
decCheckConfig [--nodb]
-a    checks all DEC configuration
-e    checks entities configuration in dec_interfaces.xml
--nodb no Inventory checks
-i    checks incoming file-types configured in dec_incoming_files.xml
-o    checks outgoing file-types configured in dec_outgoing_files.xml
-m    checks the mail configuration placed in ft_mail_config.xml
-t    checks the In-Trays configuration placed in dec_incoming_files.xml
-l    checks the log configuration
-h    it shows the help of the tool
-u    it shows the usage of the tool
-v    it shows the version number
-V    it performs the execution in Verbose mode
-D    it performs the execution in Debug mode     

== Author
DEIMOS-Space S.L.

== Copyright
Copyright (c) 2005 ESA - DEIMOS Space S.L.

\end{verbatim}


\section{decCheckSent} \label{decCheckSent}

\begin{verbatim}
== Synopsis

This is a DEC command line tool that lists the contents of the <Upload> directory
corresponding to a given interface configured in interfaces.xml

== Usage
decCheckSent -m <Interface_Name> [-t]
--mnemonic  <MNEMONIC> (mnemonic is case sensitive)
--temp      it shows the content of the <UploadTemp> directory
--Show      it shows all available I/Fs registered in the Inventory
--help      shows this help
--usage     shows the usage
--Debug     shows Debug info during the execution
--version   shows version number

== Author
Deimos-Space S.L.

== Copyright
Copyright (c) 2006 ESA - Deimos Space S.L.
\end{verbatim}

\section{decConfigInterface2DB}

\begin{verbatim}
== Synopsis

This is a Data Exchange Component command line tool that synchronizes the Entities configuration file
with DEC Inventory. It extracts all the I/Fs from the dec_interfaces.xml file and 
inserts them in the DEC Inventory.

As well it allows to specify a new I/F mnemonic to be loaded into the DEC Inventory with 
the "--add" command line option.

== Usage
decConfigInterfaceDB --add <MNEMONIC> | --process EXTERNAL
--add <MNEMONIC>     (mnemonic is case sensitive) add the specified Entity  
--process EXTERNAL   process $DEC_CONFIG/dec_interfaces.xml
--Show               it shows all I/Fs already loaded in the DCC Inventory
--Verbose            execution in verbose mode
--version            shows version number
--help      shows this help
--usage     shows the usage

== Author
Deimos-Space S.L.

== Copyright
Copyright (c) 2006 ESA - Deimos Space S.L.

\end{verbatim}

\section{decDeliverFiles}

\section{decGetFiles4Transfer}

\begin{verbatim}

== Synopsis

This is a DEC command line tool that retrieves files to be transferred (push)
from a source directory specified by $DEC_DELIVERY_ROOT environment variable.

Files present in the source directory are filtered according to the rules 
defined in ft_outgoing_files.xml
=> wildcards such as  <File Type="S2A*">
=> children directories, such as <File Type="GIP_PROBA2"> 

Files gathered are finally placed into the directory defined in the 
configuration <GlobalOutbox> present in dec_config.xml 

-O flag:
The "ONCE" flag registers in the Inventory all the files sent. As well it checks
prior to the delivery whether a files has been previously sent or not to avoid 
delivering it twice to the same Interface.

== Usage
decGetFiles4Transfer [-O] [-l]
--ONCE      The file is just sent once for each I/F
--list      list only
--help      shows this help
--usage     shows the usage
--Debug     shows Debug info during the execution
--version   shows version number

== Author
Deimos-Space S.L. (bolf)

== Copyright
Copyright (c) 2006 ESA - Deimos Space S.L.


\end{verbatim}

\section{decGetFromInterface}

\begin{verbatim}

decGetFromInterface 

== Synopsis

This is a DEC command line tool that polls the I/Fs for retrieving 
files of registered filetypes. As well It retrieves the I/F 
exchange directory file content linked to a time-stamp.

-l flag:

With this option, only "List" of new availables files for Retrieving and Tracking is done.
This flag overrides configuration flags RegisterContentFlag RetrieveContentFlag in dec_interfaces.xml
So Check ONLY of new Files is performed anyway.

-R flag:

With this option (Reporting), DEC Reports will be created (see dcc_config.xml). 
Report files are initally placed in the Interface local inbox and
if configured in files2InTrays.xml disseminated as nominal retrieved file.

--del-unknown:

It overrides the dcc_config.xml configuration parameter DeleteUnknown and explicitly
commands for removal of unknown files not configured in ft_incoming_files.xml


== Usage
decGetFromInterface -m <MNEMONIC>  [-l] [--nodb]
--mnemonic  <MNEMONIC> (mnemonic is case sensitive)
--list      list only (not downloading and no ingestion)
--nodb      no Inventory recording
--no-intray skip step of delivery to intrays
--del-unknown it deletes remote files not configured in ft_incoming_files.xmls
--receipt   create only receipt file-list with the content available
--Report    create a Report when new files have been retrieved
--Show      it shows all available I/Fs registered in the DEC Inventory
--help      shows this help
--usage     shows the usage
--Debug     shows Debug info during the execution
--Unknown   shows Unknown files
--Benchmark shows Benchmark info during the execution
--version   shows version number

== Author
DEIMOS-Space S.L.

\end{verbatim}


\section{decODataClient}
\label{decODataClient}

\begin{verbatim}
The decODataClient is a simple OData client to hit Copernicus Sentinel
servers to extract metadata and download the associated products.

DHUS queries:

Query by Sentinel-2 Datatake Identifier: which looks for the associated
published Complete Single Tile products 

--query dhus:s1:S1B
--query dhus:s2:S2A
--query dhus:gnss:S1B
--query dhus:s2:GS2B_20200903T104429_018252_N02.14
--query dhus:s1:S1A:S1A_EW_GRDM
--query dhus:s1:S1A:45C1F

--query adgs:s2:S2B

--query dhus_s5p:s5:S5

PRIP: queries:

Query is so far targetting S2PRIP specifically.

--query prip:S2B_OPER_MSI_L0__DS_VGS2_20201028T163228_S20201028T144729_N02.09:MSI_L0__GR

The queries can be done also specifically by PDI:
decODataClient -u <user> -p <pass> -q prip:S2B_OPER_MSI_L0__DS_VGS1_20210126T095435_S20210126T080454_N02.09 -V

== Usage
decODataClient 
--user      <username>
--password  <password>
--time       2021-03-24T00:00:00.000 (DHuS format applied to the IngestionDate)
--delay     <hour delay> (decimal offset 0.5-0.25 is scaled to minutes)
--sensing   "2021-03-16T00:00:00.000,2021-03-20T00:00:00.000"
--creation  "2021-03-16T00:00:00.000,2021-03-20T00:00:00.000"
--H         <hours delay> (decimal offset 0.5-0.25 is scaled to minutes)
--Location  <full_path_dir>
--format    json | xml | csv
\end{verbatim}	
	
\section{decListener}

\begin{verbatim}

== Synopsis

This is a Data Exchange Component (DEC) command line tool 
that manages the I/Fs listeners for data retrieval.

The DEC listeners automates the file pulling from the configured interface.
One listener is devoted for every interface configured.

The behaviour of the listener is driven by the settings defined
in the configuration file $DEC_CONFIG/dec_interfaces.xml

Alternatively the listener settings can be overriden with the command line options.

== Usage
decListener  --all [-R]| --mnemonic <MNEMONIC> --interval <seconds>
--all                 starts a listener for each I/Fs
--Reload              force a Restart of all listeners
--stop <MNEMONIC>     it stops of the listener for the given I/F
--Stop                it stops of all listeners
--check               it checks whether the listeners are running
--mnemonic <MNEMONIC> (mnemonic is case sensitive)
--interval            the frequency it is polled I/F given by MNEMONIC (in seconds)
--nodb                no Inventory recording
--no-intray           skip step of delivery to intrays upon download
--help                shows this help
--Debug               shows Debug info during the execution
--version             shows version number      

== Author
DEIMOS-Space S.L.

== Copyright
Copyright (c) 2005 ESA - DEIMOS Space S.L.

\end{verbatim}

\section{decManageDB}

\begin{verbatim}

== Usage
decManageDB --create-tables | --drop-tables
--create-tables   create all minarc required tables
--drop-tables     drops all minarc tables
--rpf             selector to include reference planning tables
--Debug           shows Debug info during the execution
--help            shows this help


\end{verbatim}

\label{decNotify2Interface}
\section{decNotify2Interface}

\begin{verbatim}

== Usage
decNotify2Interface -m <MNEMONIC> --OK | --KO
--mnemonic  <MNEMONIC> (mnemonic is case sensitive)
--OK        notify success in the delivery to the I/F -f full_path_filelist list of files send
--KO        notify failure in the delivery to the I/F -f full_path_filelist list of files failed to be sent
--help      shows this help
--usage     shows the usage
--Debug     shows Debug info during the execution
--version   shows version number

\end{verbatim}

\section{decSend2Interface}

\begin{verbatim}

== Synopsis

This is a DEC command line tool that deliver files to a given I/F in PUSH mode.
It delivers files using the configured protocols (s)ftp and email. 
Files sent can be registered in an Inventory and the delivery date is set to the latest one.

This command can be used in order to send a given file just once 
(for each delivery method: ftp, email) for a given Interface. 
Use "-O" flag to enable this behaviour.

-R flag:

With this option (Report), a Report "List" with the new files sent is created. 
This Report file is initally placed in the Interface local inbox.


== Usage
decSend2Interface -m <MNEMONIC> [-O] [--nodb]
--mnemonic  <MNEMONIC> (mnemonic is case sensitive)
--ONCE      The file is just sent once for that I/F
--AUTO      local outbox Automatic management 
--loops <n> n is the number of Loop retries to achieve the Delivery
--delay <s> s seconds of delay between each Loop Retry
[60 secs by default if it is not specified]
--retries <r>  r is the number of retries on each Loop for each file
--Report    create a Report with the list of files delivered to the Interface
--list      list only (not downloading and no ingestion)
--Nomail    avoids mail notification to the I/F after successfully delivery
--Show      it shows all available I/Fs registered in the Inventory
--nodb      no usage of the Inventory for recording operations
--help      shows this help
--usage     shows the usage
--Debug     shows Debug info during the execution
--version   shows version number

== Author
Deimos-Space S.L.

== Copyright
Copyright (c) 2006 ESA - Deimos Space S.L.


\end{verbatim}

\section{decSmokeTests}

\label{decStats}
\section{decStats}

\begin{verbatim}

== Synopsis

This is a DEC/DCC command line tool that shows file reception statistics  

== Usage
decStats
--Hours <hours>            status of last n hours
--file <filename>          status of a given filename
--help                     shows this help
--usage                    shows the usage
--Debug                    shows Debug info during the execution
--version                  shows version number


== Author
DEIMOS-Space S.L.

== Copyright
Copyright (c) 2008 ESA - DEIMOS Space S.L.

\end{verbatim}

\section{decUnitTests}

\section{decUnitTests\_IERS}

\section{decValidateConfig}

\begin{verbatim}

== Synopsis

This is a DEC command line tool that checks the validity of DEC configuration
files according to DEC's XSD schemas. This tool should be run everytime a 
configuration change is performed.

-e flag:

With this option the Interfaces (Entities) configuration file (interfaces.xml)
is validated using the schema interfaces.xsd

-g flag:

With the main DEC configuration file (dec_config.xml)
is validated using the schema dec_config.xsd

-i flag:

With this option the Incoming file-types configuration file (dec_incoming_files.xml)
is validated using the schema dec_incoming_files.xsd

-o flag:

With this option the Outgoing file-types configuration file (dec_outgoing_files.xml)
is validated using the schema dec_outgoing_files.xsd

-m flag:

With this option the DEC Mail configuration file (ft_mail_config.xml) is
validated using the schema ft_mail_config.xsd

-l flag:

With this option the DEC Logs configuration file (dec_log_config.xml) is
validated using the schema dec_log_config.xsd

-a flag:

This is the all flag, which performs all the checks described before.


== Usage
-a     Check all DEC configuration files
-g     Check DEC's general configuration file dec_config.xml
-e     Check the Entities Configuration file dec_interfaces.xml
-m     Check the mail configuration file ft_mail_config.xml
-i     Check the incoming file-types configuration file dec_incoming_files.xml
-i     Check the outgoing file-types configuration file dec_outgoing_files.xml
-h     shows this help
-v     shows version number


== Author
DEIMOS-Space S.L.

== Copyright
Copyright (c) 2007 ESA - DEIMOS Space S.L.

\end{verbatim}

\end{document}

