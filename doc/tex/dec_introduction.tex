%
% Elecnor Deimos
%
% Data Exchange Component
%
% Borja Lopez Fernandez (BOLF)
%
% DEC Introduction
%

\documentclass[dec_sum_main.tex]{subfiles}
 
\begin{document}

\section{Purpose \& Scope}
The Data Exchange Component (DEC) is a SW helper to gather, transform, circulate, and archive files autonomously amongst different interfaces. \newline
\par
\noindent
The scope of DEC SW usually lies on the different ICD defined for communication and exchange of data (generally \textit{files}). As such, it relies on different COTS to delegate the implementation of the different protocols supported.

\section{Design Drivers}
This section enumerates a high level overview for the main design drivers that allows DEC to interface efficiently for the configured exchanges:

\begin{itemize}
	\item \textit{Flexibility} : the ability to \textit{pull} \& \textit{push} files from configurable interfaces and configurable circulation rules
	\item \textit{Robustness} : this is to perform “atomic” operations during file circulation ; the state for each operation is always known being network errors tolerant
	\item \textit{Performance} : parallelisation of the circulation to fruit the available network bandwidth, support of file compression mechanisms to reduce the footprint of transfers and local disseminations for which duplication of files by \textit{cp} or \textit{mv} can be avoided by usage of \textit{hardlinks}
	\item \textit{Resiliency} : to \textit{autonomously} recover from network errors, downtime and eventual glitches
\end{itemize}

\section{Features}
This section enumerates some of the main high-level features offered by DEC.

\subsection{Pull Circulations}
The DEC SW offers the capacity to \textit{autonomously} pull files from the different configuration interfaces for which the frequency of polling can be configured for each interface to adapt for the workflows settled between the suppliers and the consumers.
\par
\noindent


\subsection{Push Circulations}
The DEC SW offers the capacity and command line interfaces to \textit{push} files towards the different configured interfaces.

It is as well possible to automate \textit{push} transfers on regular basis with configurable frequency for every interface.

\section{Download DEC SW}
This section contains the links to download the DEC SW and the \textit{Gemfile} with the definition of the \textit{gem} dependencies. It is necessary to be logged-in with your DEIMOS \textit{gmail} account to \textit{authorise} the download.

\par
\noindent

\begin{itemize}
	\item \href{https://drive.google.com/uc?export=download&id=18TAD2BAJpHzgBdZGf3W8KQZCY7SPpKbU}{\textit{Gemfile}} : file with the \textit{gem} dependencies 
	\item \href{https://drive.google.com/uc?export=download&id=1gieDRpDEBzKv5Xr0qwPvBz34RHphTnQq}{\textit{dec-stable.gem}} : DEC gem installer
	\item\href{https://drive.google.com/uc?export=download&id=1JBAnxdRwaPK8PmrxuRi3qea43NOPRJbK/view?usp=sharing}{\textit{dec\_test.bash}} : definition of the environment variables used by the unit tests for bash console 
	\item\href{https://drive.google.com/uc?export=download&id=1HHWOVxCZbY6Surv_kKyF5kGTy45CvUKl/view?usp=sharing}{\textit{dec\_test.env}} : definition of the environment variables used by the unit tests for docker container parametrisation
\end{itemize}

\end{document}

